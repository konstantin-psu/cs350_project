\documentclass{IEEEtran}
\usepackage[table,xcdraw]{xcolor}
\usepackage{float}
\usepackage{tikz}
\usepackage{cite}
\usepackage{graphicx}
\usepackage{pgfplots}
\usepackage{adjustbox}
\usepackage{url}
\usepackage[plain]{algorithm}
\usepackage{algpseudocode}
\usepackage{amsmath}

\usepackage{listings}
\usepackage{color}

\definecolor{dkgreen}{rgb}{0,0.6,0}
\definecolor{gray}{rgb}{0.5,0.5,0.5}
\definecolor{mauve}{rgb}{0.58,0,0.82}

\definecolor{bblue}{HTML}{4F81BD}
\definecolor{rred}{HTML}{C0504D}
\definecolor{ggreen}{HTML}{9BBB59}
\definecolor{ppurple}{HTML}{9F4C7C}
\pgfplotsset{compat=newest}

\lstset{frame=tb,
    language=Java,
    aboveskip=3mm,
    belowskip=3mm,
    showstringspaces=false,
    columns=flexible,
    basicstyle={\small\ttfamily},
    numbers=none,
    numberstyle=\tiny\color{gray},
    keywordstyle=\color{blue},
    commentstyle=\color{dkgreen},
    stringstyle=\color{mauve},
    breaklines=true,
    breakatwhitespace=true,
    tabsize=3,
    xleftmargin=1cm,
    xrightmargin=1cm
}
 
\title{Sorting}
\author{
    Melanie Marks, Konstantin Macarenco\\
}

\begin{document}
\maketitle
\thispagestyle{plain}
\pagestyle{plain}
\begin{abstract}

\end{abstract}
\tableofcontents
\section{Introduction}\label{sec:intro}
Citation example\cite{ABOOK}


\pagebreak
\section{Implementation}\label{sec:Implementation}
We had originally planed to write our implementations in C++ and our test programs in bash and python.  \\
choosing a low-overhead language won't necessarily produce "more accurate" results.  The absolute costs of the basic operations are irrelevant to the purpose of the project, which is to compare analytical and measured results.


\begin{enumerate}
\item Quicksort

\begin{enumerate}
\item Original
\item 3-Way Partition
\item Dual-Pivot
\item Lomuto (in-place)
\item Hoare (in-place)
\item Dual-Pivot (in-place)
\end{enumerate}

\end{enumerate}





\begin{algorithm}[h!]
    \begin{algorithmic}[1]
        \Function{Mergesort} {input:  an array of numbers A, output: a sorted array}
            \If {$len(A) \ge 1$} %\\
                \State{} \Return {A} // because it is sorted
            \EndIf
            \State{$start \leftarrow$ 0}
            \State{$end \leftarrow len(A) - 1$}
            \State{$middle \leftarrow \cfrac{start + (end - start)}{2}$}
            \State{$l \leftarrow mergesort(A[start \to (middle + 1)])$}
            \State{$r \leftarrow mergesort(A[(middle + 1) \to (end + 1)])$}
            \State{}\Return {$merge(l, r)$}
        \EndFunction{}
    \end{algorithmic}
\end{algorithm}



\begin{enumerate}
\item Mergesort
\begin{enumerate}
\item Original
\item Bottom-Up
\item With Insertion Sort
\item Checking If Sorted
\end{enumerate}

\end{enumerate}


\begin{algorithm}[h!]
    \begin{algorithmic}[1]
        \Function{QuickSortThreeWay} {input:  an array of numbers $A$, output: a sorted array}
            \State{$pivot = get_pivot()$}
            \State{$less=[]$}  //To store all values less than the pivot
            \State{$greater=[]$}  //To store all values greater than the pivot
            \State{$equal=[]$}    //Store all values equal to the pivot
            \For{$x$ in $A$}
                \If{$x < pivot$}
                    \State{$less.append(x)$}
                \ElsIf{$x > pivot$}
                    \State{$greater.append(x)$}
                \Else{ $x == pivot$}
                    \State{$equal.append(x)$}
                \EndIf
            \EndFor{}  
            \State{$less = QuickSortThreeWay(less)$}
            \State{$greater = QuickSortThreeWay(greater)$}
            \State{} \Return{$concatenate(less, equal, greater)$}
 
        \EndFunction{}
    \end{algorithmic}
\end{algorithm}


\begin{algorithm}[h!]
    \begin{algorithmic}[1]
        \Function{QuickSortDualPivot} {input:  an array of numbers $A$, output: a sorted array}
            \State{$pivot1 = get_pivot()$}
            \State{$pivot2 = get_pivot()$}
            \State{$less=[]$}  //To store all values less than the pivot
            \State{$greater=[]$}  //To store all values greater than the pivot
            \State{$equal=[]$}    //Store all values equal to the pivot
            \For{$x$ in $A$}
                \If{$x < pivot$}
                    \State{$less.append(x)$}
                \ElsIf{$x > pivot$}
                    \State{$greater.append(x)$}
                \Else{ $x == pivot$}
                    \State{$equal.append(x)$}
                \EndIf
            \EndFor{}  
            \State{$less = QuickSortThreeWay(less)$}
            \State{$greater = QuickSortThreeWay(greater)$}
            \State{} \Return{$concatenate(less, equal, greater)$}
 
        \EndFunction{}
    \end{algorithmic}
\end{algorithm}

\begin{algorithm}[h!]
    \begin{algorithmic}[1]
        \Function{QuickSort} {input:  an array of numbers A, output: a sorted array}
        \EndFunction{}
    \end{algorithmic}
\end{algorithm}

\begin{algorithm}[h!]
    \begin{algorithmic}[1]
        \Function{QuickSort} {input:  an array of numbers A, output: a sorted array}
        \EndFunction{}
    \end{algorithmic}
\end{algorithm}


\begin{enumerate}
\item heapsort

\begin{enumerate}
\item Original 
\item Bottom-up
\item Smoothsort\\
Since heapsort uses binary trees, it is structured to have the largest element on the leftmost side of the array.  To mo
heap sort structures the max-heap to have the largest element of the heap on the leftmost side of the array.
\item Skewsort
\end{enumerate}

\end{enumerate}




\begin{enumerate}
\item changing implementation language choice from C/C++ to python
\item creating randomized testing data and issues associated
\begin{enumerate}
\item space for original and improved versions of the implementations
\item speed
\end{enumerate}
\end{enumerate}


\section{Testing}\label{sec:testing}

\textbf{Testing: conclusions}

\section{General Conclusion}

\noindent
From the rubric:\\
\textbf{Can understand and implement standard algorithms}\\
Describes or depicts algorithms using examples\\
Discusses implementation issues\\
\\
\textbf{Can write programs that are understandable and are algorithmically sound}\\
Program fragments are presented\\
... are understandable\\
... are algorithmically sound\\
Sound measurement technique\\
Generation of experimental data sets\\
Testing for correct results\\
\\
\textbf{Makes connections between implementation and complexity theory}\\
Compares measured and predicted performance\\
Explains discrepancies, if any\\
Discusses why an asymptotically inferior algorithm might perform better\\
\\
\textbf{Demonstrates initiative, originality, and algorithmic insights}\\
Initiative\\
Originality\\
Algorithmic insights\\
\\
\textbf{Document Communicates clearly}\\
Document is concise \\
Experimental procedure is described \\
Language is clear and correct \\
Experimental results presented \\
Purpose of project is described \\
Conclusions Presented \\

\pagebreak
\bibliographystyle{ieeetr}
\bibliography{bibliography}
\end{document}






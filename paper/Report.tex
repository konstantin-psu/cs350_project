\documentclass{IEEEtran}
\usepackage[table,xcdraw]{xcolor}
\usepackage{float}
\usepackage{tikz}
\usepackage{cite}
\usepackage{graphicx}
\usepackage{pgfplots}
\usepackage{adjustbox}
\usepackage{url}

\usepackage{listings}
\usepackage{color}

\definecolor{dkgreen}{rgb}{0,0.6,0}
\definecolor{gray}{rgb}{0.5,0.5,0.5}
\definecolor{mauve}{rgb}{0.58,0,0.82}

\definecolor{bblue}{HTML}{4F81BD}
\definecolor{rred}{HTML}{C0504D}
\definecolor{ggreen}{HTML}{9BBB59}
\definecolor{ppurple}{HTML}{9F4C7C}
\pgfplotsset{compat=newest}

\lstset{frame=tb,
    language=Java,
    aboveskip=3mm,
    belowskip=3mm,
    showstringspaces=false,
    columns=flexible,
    basicstyle={\small\ttfamily},
    numbers=none,
    numberstyle=\tiny\color{gray},
    keywordstyle=\color{blue},
    commentstyle=\color{dkgreen},
    stringstyle=\color{mauve},
    breaklines=true,
    breakatwhitespace=true,
    tabsize=3,
    xleftmargin=1cm,
    xrightmargin=1cm
}
 
\title{Sorting}
\author{
    Melanie Marks, Konstantin Macarenco\\
}

\begin{document}
\maketitle
\thispagestyle{plain}
\pagestyle{plain}
\begin{abstract}

\end{abstract}
\tableofcontents
\section{Introduction}\label{sec:intro}
Citation example\cite{ABOOK}

\pagebreak
\section{Implementation}\label{sec:Implementation}
\begin{enumerate}
\item decided to focus on quicksort, mergesort, and heapsort and improvements associated with those algorithms
\item changing implementation language choice from C/C++ to python
\begin{enumerate}
\item professor's suggestion in feedback from project proposal
\end{enumerate}
\item creating randomized testing data and issues associated
\begin{enumerate}
\item space for original and improved versions of the implementations
\item speed
\end{enumerate}
\end{enumerate}

\pagebreak
\section{Testing}\label{sec:testing}

\textbf{Testing: conclusions}


\pagebreak
\section{General Conclusion}


\pagebreak
\noindent
From the rubric:\\
\textbf{Can understand and implement standard algorithms}\\
Describes or depicts algorithms using examples\\
Discusses implementation issues\\
\\
\textbf{Can write programs that are understandable and are algorithmically sound}\\
Program fragments are presented\\
... are understandable\\
... are algorithmically sound\\
Sound measurement technique\\
Generation of experimental data sets\\
Testing for correct results\\
\\
\textbf{Makes connections between implementation and complexity theory}\\
Compares measured and predicted performance\\
Explains discrepancies, if any\\
Discusses why an asymptotically inferior algorithm might perform better\\
\\
\textbf{Demonstrates initiative, originality, and algorithmic insights}\\
Initiative\\
Originality\\
Algorithmic insights\\
\\
\textbf{Document Communicates clearly}\\
Document is concise \\
Experimental procedure is described \\
Language is clear and correct \\
Experimental results presented \\
Purpose of project is described \\
Conclusions Presented \\

\pagebreak
\bibliographystyle{ieeetr}
\bibliography{bibliography}
\end{document}






\documentclass{article}
\usepackage[utf8]{inputenc}
\usepackage{fullpage}

\title{CS 350 Project Proposal}
\author{Konstantin Macarenco and Melanie Marks }
\date{February 2015}

\begin{document}

\maketitle


%Your proposal should identify:
%
%the name(s) of the student(s) working on the project;
%the choice of project topic;
%the implementation language;
%a list of the specific features that you expect to include in your final report; and
%a time plan that identifies at least 3 specific goals for each of the remaining 3 weeks of the term.
%A collaboration plan that describes how your team intends to work together.  


\noindent
\section*{Topic Choice:}

Good Sorts\\
Lots of options for algorithms.  More optimization options (pivot points, partitions, ...).  Good to learn what optimizations make the biggest differences.\\
%expectation: different optimization are more suitable for different inputs
\section*{Implementation Language:}

We chose C and C++ as the languages for implementing the sorting algorithms because of their speed and low overhead, even though C/C++ is harder to implement, it will produce more accurate results. \\
On the other hand C is not well suitable as a test harness, so we chose Python and Bash for test automatization.\\
%and bash for running those tests on our sorting implementations.\\

To verify correctness for gathered results, and effectively measure execution run-time, various methods will be used, such as instrumenting code, profiling, comparison of estimated against actual run-time.\\ 
%Profiling tools such as "gprof"\\
\section*{Expected (Possible) Features:} 
\begin{itemize}
\item (Quicksort) Find the crossover point (for comparing to less efficient algorithms)
\item (Quicksort) The impact of different pivot choosing methods - first vs middle vs random.
\item The impact of different partitioning algorithms - Lomuto vs Hoare.
\item visual representation of results
\item Visual comparisons of each of the sorting algorithms
\item Compare the space requirements for each algorithm
\item bottom-up vs top-down mergesort
\item Verification of results (should be simple).
\begin{itemize}
\item language or system sort comparison
\end{itemize}
\item If time allows: 
\begin{itemize}
%\item Java vs C++, i.e virtual machine vs low level.
\item Multithreaded vs single threaded.
%TODO all tests, for consistency, should be performed on the same machine, maybe request a dedicated time for one of the linuxlab machines?
\item GPU
\end{itemize}
\end{itemize}
\section*{Specific Goals:} 
\begin{enumerate}
\item Week 1  Implementation\\
Implement the quicksort, mergesort, and heapsort algorithms.\\
Research improvements for mergesort and heapsort\\
Write algorithms in our own words for the report\\
estimate running times (summation formulas) and compare to actual times\\
Make note of any interesting implementation details or implemented algorithms not in the class text.
\item Week 2  Testing \\
Create test programs
\begin{itemize}
\item sorted
\item reverse sorted
\item randomized
\item 25\% sorted
\item 85\% sorted
\item gaussian distribution
\item list of the same number
\item test stability
\item more if time allows
\end{itemize}
Run tests\\
Implement various improvements.\\
\begin{itemize}
\item (Mergesort) dividing data into three sections and sorting those separately before merging
\item (heapsort) multiple heaps
\end{itemize}
Test those improvements and document any changes in the algorithm's behavior and how their space requirements compare.\\
Make note of unexpected results, clever improvements, .. 
\item Week 3  Report \\
gather notes, create graphs and other graphics\\
\end{enumerate}
\section*{Collaboration Plan: }
Along with meeting in person after class on Tuesdays and Saturday afternoons, we have also created a shared github repository for our code and are using ShareLatex for our proposal, weekly notes, and final project report.\\
\end{document}
